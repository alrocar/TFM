%% Generated by Sphinx.
\def\sphinxdocclass{report}
\documentclass[letterpaper,10pt,spanish]{sphinxmanual}
\ifdefined\pdfpxdimen
   \let\sphinxpxdimen\pdfpxdimen\else\newdimen\sphinxpxdimen
\fi \sphinxpxdimen=.75bp\relax

\usepackage[utf8]{inputenc}
\ifdefined\DeclareUnicodeCharacter
 \ifdefined\DeclareUnicodeCharacterAsOptional
  \DeclareUnicodeCharacter{"00A0}{\nobreakspace}
  \DeclareUnicodeCharacter{"2500}{\sphinxunichar{2500}}
  \DeclareUnicodeCharacter{"2502}{\sphinxunichar{2502}}
  \DeclareUnicodeCharacter{"2514}{\sphinxunichar{2514}}
  \DeclareUnicodeCharacter{"251C}{\sphinxunichar{251C}}
  \DeclareUnicodeCharacter{"2572}{\textbackslash}
 \else
  \DeclareUnicodeCharacter{00A0}{\nobreakspace}
  \DeclareUnicodeCharacter{2500}{\sphinxunichar{2500}}
  \DeclareUnicodeCharacter{2502}{\sphinxunichar{2502}}
  \DeclareUnicodeCharacter{2514}{\sphinxunichar{2514}}
  \DeclareUnicodeCharacter{251C}{\sphinxunichar{251C}}
  \DeclareUnicodeCharacter{2572}{\textbackslash}
 \fi
\fi
\usepackage{cmap}
\usepackage[T1]{fontenc}
\usepackage{amsmath,amssymb,amstext}
\usepackage{babel}
\usepackage{times}
\usepackage[Sonny]{fncychap}
\usepackage[dontkeepoldnames]{sphinx}

\usepackage{geometry}

% Include hyperref last.
\usepackage{hyperref}
% Fix anchor placement for figures with captions.
\usepackage{hypcap}% it must be loaded after hyperref.
% Set up styles of URL: it should be placed after hyperref.
\urlstyle{same}
\addto\captionsspanish{\renewcommand{\contentsname}{Contents:}}

\addto\captionsspanish{\renewcommand{\figurename}{Figura}}
\addto\captionsspanish{\renewcommand{\tablename}{Tabla}}
\addto\captionsspanish{\renewcommand{\literalblockname}{Lista}}

\addto\captionsspanish{\renewcommand{\literalblockcontinuedname}{continued from previous page}}
\addto\captionsspanish{\renewcommand{\literalblockcontinuesname}{continues on next page}}

\addto\extrasspanish{\def\pageautorefname{página}}

\setcounter{tocdepth}{0}



\title{CARTO Big Data connectors}
\date{01 de octubre de 2017}
\release{1.0.0}
\author{Alberto Romeu Carrasco}
\newcommand{\sphinxlogo}{\vbox{}}
\renewcommand{\releasename}{Versión}
\makeindex

\begin{document}
\ifnum\catcode`\"=\active\shorthandoff{"}\fi
\maketitle
\sphinxtableofcontents
\phantomsection\label{\detokenize{index::doc}}



\chapter{Resumen ejecutivo}
\label{\detokenize{tfm/01-abstract::doc}}\label{\detokenize{tfm/01-abstract:resumen-ejecutivo}}\label{\detokenize{tfm/01-abstract:carto-big-data-connectors}}
CARTO %
\begin{footnote}[1]\sphinxAtStartFootnote
\sphinxurl{https://www.carto.com}
%
\end{footnote} es una plataforma de \sphinxstyleemphasis{Location Intelligence} %
\begin{footnote}[2]\sphinxAtStartFootnote
{\hyperref[\detokenize{tfm/99-glosario:location-intelligence}]{\sphinxcrossref{\DUrole{std,std-ref}{Location Intelligence}}}} Añadir directamente aquí la definición
%
\end{footnote} que permite transformar datos geoespaciales en resultados de negocio.

Actualmente, las organizaciones que utilizan \sphinxstylestrong{CARTO} como herramienta de análisis geoespacial tienen multitud de fuentes de información y aplicaciones ya instaladas que generan continuamente nuevos datos.

El principal valor de \sphinxstylestrong{CARTO} para estas organizaciones es el de conectarse con esas fuentes de información (\sphinxstyleemphasis{datos de CRM, ERPs, hojas de cálculo, archivos con contenido geoespacial, bases de datos relacionales, etc.}) a través de una interfaz sencilla e intuitiva, para generar nueva información de valor añadido para su negocio mediante análisis geoespaciales y visualizaciones.

En determinadas organizaciones, especialmente de tamaño medio o grande, ocurre que diversos equipos gestionan sus datos con sistemas de información heterogéneos que utilizan repositorios de datos tales como:
\begin{itemize}
\item {} 
Hojas de cálculo y archivos CSV

\item {} 
Documentos de Google Drive

\item {} 
CRMs tales como Salesforce

\item {} 
Herramientas de marketing digital como Mailchimp

\item {} 
Servidores de datos geoespaciales como ArcGIS

\item {} 
Bases de datos operacionales (relacionales o NoSQL)

\item {} 
Sistemas de ficheros distribuidos HDFS

\item {} 
Otros sistemas (Twitter, Dropbox, Instagram, etc.)

\end{itemize}

Estas organizaciones utilizan \sphinxstylestrong{CARTO} para importar sus datos y analizar la información que generan los distintos departamentos de manera conjunta, respondiendo a sus preguntas de negocio y encontrando además respuesta a otras preguntas que no se habían planteado en un principio.

\sphinxstylestrong{CARTO} cuenta con la posibilidad de importar datos desde diversas fuentes de datos, pero carece de soporte nativo para conectar a sistemas de almacenamiento masivo de datos usados generalmente para almacenar datos operacionales y resultados agregados obtenidos por los departamentos de \sphinxstyleemphasis{data science}.

El objetivo de este trabajo final de máster consiste en el desarrollo de conectores para \sphinxstylestrong{CARTO} que permitan incluir en los cuadros de mandos (\sphinxstyleemphasis{dashboards}), información proveniente de sistemas de almacenamiento masivo tales como: HDFS, Hive, Impala, MongoDB, Cassandra, BigQuery u otros.

Palabras clave: \sphinxstyleemphasis{BASH, Docker, Vagrant, Location Intelligence, AWS, HDFS, Hadoop, BigQuery, Hive, Impala, Spark, NoSQL, Cassandra, MongoDB, CARTO, dashboards, análisis geoespacial}


\chapter{Glosario}
\label{\detokenize{tfm/99-glosario::doc}}\label{\detokenize{tfm/99-glosario:glosario}}

\section{CRM}
\label{\detokenize{tfm/99-glosario:crm}}

\section{CSV}
\label{\detokenize{tfm/99-glosario:csv}}

\section{ERP}
\label{\detokenize{tfm/99-glosario:erp}}

\section{HDFS}
\label{\detokenize{tfm/99-glosario:hdfs}}

\section{Location Intelligence}
\label{\detokenize{tfm/99-glosario:location-intelligence}}\label{\detokenize{tfm/99-glosario:id1}}

\section{NoSQL}
\label{\detokenize{tfm/99-glosario:nosql}}

\chapter{Indices and tables}
\label{\detokenize{index:indices-and-tables}}\begin{itemize}
\item {} 
\DUrole{xref,std,std-ref}{genindex}

\item {} 
\DUrole{xref,std,std-ref}{modindex}

\item {} 
\DUrole{xref,std,std-ref}{search}

\end{itemize}



\renewcommand{\indexname}{Índice}
\printindex
\end{document}